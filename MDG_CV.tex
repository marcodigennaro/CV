
\documentclass[10pt,a4paper,ragged2e,withhyper]{altacv}

\geometry{
	left=1.25cm,
	right=1.25cm,
	top=1.3cm,
	bottom=1.2cm,
	columnsep=1.cm}

% The paracol package lets you typeset columns of text in parallel
\usepackage{paracol}
\usepackage{fancyhdr}
\usepackage{eurosym}

% Change the font if you want to, depending on whether
% you're using pdflatex or xelatex/lualatex
\ifxetexorluatex
  % If using xelatex or lualatex:
  \setmainfont{Roboto Slab}
  \setsansfont{Lato}
  \renewcommand{\familydefault}{\sfdefault}
\else
  % If using pdflatex:
  \usepackage[rm]{roboto}
  \usepackage[defaultsans]{lato}
  % \usepackage{sourcesanspro}
  \renewcommand{\familydefault}{\sfdefault}
\fi

% Change the colours if you want to
\colorlet{accent}{blue!70!black}
\colorlet{heading}{blue!70!black}
\colorlet{emphasis}{black}
\colorlet{body}{black!80!white}

% Change some fonts, if necessary
\renewcommand{\namefont}{\Huge\rmfamily\bfseries}
\renewcommand{\personalinfofont}{\footnotesize}
\renewcommand{\cvsectionfont}{\Large\rmfamily\bfseries}
\renewcommand{\cvsubsectionfont}{\large\bfseries}


% Change the bullets for itemize and rating marker
% for \cvskill if you want to
%\renewcommand{\itemmarker}{{\small\textbullet}}
%\renewcommand{\ratingmarker}{\faCircle}

\begin{document}

\name{\bigskip Marco Di Gennaro | PhD\smallskip}
%\tagline{Quantum + Data + Materials Science = Materials Informatics | Ph.D. \smallskip}
\tagline{Freelance Data Scientist with >10 years of R\&D experience}

\personalinfo{%
\bigskip
\begin{tabular}{p{3cm}p{4.5cm}p{3cm}}
\location{Bruxelles (BE)}  &
\email{mdg@atomistic-modelling.com} &
%\faResearchgate \href{https://www.researchgate.net/profile/Marco_Di_Gennaro}{Marco\url{_}Di\url{_}Gennaro}\\
\faLinkedin \href{https://www.linkedin.com/in/marcodig/}{marcodig}\\

\phone{+32485185559}&
\homepage{atomistic-modelling.com}&
\faGithub \href{https://github.com/marcodigennaro}{marcodigennaro} 

%\hspace{0.08cm}\raisebox{-.2\height}{\includegraphics[width=0.02\columnwidth]{orcid}}\hspace{0.1cm}\href{https://orcid.org/0000-0001-5734-5155}{orcid [0000-0001-5734-5155]} &
%\hspace{0.05cm}\raisebox{-.2\height}{\includegraphics[width=0.02\columnwidth]{X}}\hspace{0.1cm}\href{https://twitter.com/mdg_qmatinfo}{mdg\_qmatinfo} 
\end{tabular}
}

%% You can add multiple photos on the left or right
\photoR{4cm}{pic_new}

%% Depending on your tastes, you may want to make fonts of itemize environments slightly smaller
% \AtBeginEnvironment{itemize}{\small}

\makecvheader

%% Set the left/right column width ratio to 6:4.
\columnratio{0.66}

% Start a 2-column paracol. Both the left and right columns will automatically
% break across pages if things get too long.
\begin{paracol}{2}

%% LEFT COLUMN START

\cvsection{ Work Experience}

\myevent{CTO - Artificial Intelligence}{\href{https://dtsc.be/}{DTSC}}{Jun 2024 -- Present}{Bruxelles (Be)}
\begin{itemize}
\item Established the AI strategy and the network of academic collaborators
\item Led the internal projects in AI \& automatisation
\end{itemize}
\divider

\myevent{Founder \& Freelance consultant}{\href{https://atomistic-modelling.com/}{ATOM}}{Mar 2023 -- Present}{Bruxelles (Be)}
\begin{itemize}
\item Established a SME specialised in machine learning \& materials simulation
\item Derived data-driven models in material research for energy applications
\item Managed operations and strategic planning, financial strategies
%\item Cultivated partnerships within industrial and academic sectors
\item Artificial Intelligence: \href{https://github.com/marcodigennaro/LLM}{\underline{sentiment analysis through LLM}}
\item[] \href{https://github.com/marcodigennaro/windml}{\underline{Wind power production through ML}} 
\href{https://github.com/marcodigennaro/optimization}{\underline{Renewables energy optimization}} 
\item Associated Partner in the  \href{https://euspeclab.cnrs.fr/}{\underline{\emph{EUSpecLab}}}: ML techniques for  spectroscopy
\end{itemize}
\divider

\myevent{Machine Learning Scientist}{\href{https://www.toyota-europe.com/}{Toyota Motor Europe}}{Jun 2018 -- May 2023}{Bruxelles (Be)}
\begin{itemize}
\item Integrated simulations and machine learning in nano-materials research
\item Delivered data-driven decision-making in multidisciplinary research projects
\item Directed strategic research initiatives, employing Toyota’s principles 
\item Developed and distributed  several python packages for materials science
%\item Enhanced team productivity through of data-driven approach
\item Delivered training programs in python and machine learning
\item Demonstrated the use of quantum computing in material science
\item Demonstrated proficiency in securing and managing EU/HPC resources
\item Participated in the recruitment process and training of students
\end{itemize}
\divider

\myevent{Freelance Project Manager EU Projects}{\href{https://www.pin.bike/}{Pin Bike}}{Jan 2021 -- Jun 2021}{Corato (It)}
\begin{itemize}
\item \href{https://bicification-project.eu/}{\underline{Bicification}} EIT urban mobility project: pitch competition, budget negotiations ($\sim$\euro{300K}), partnership agreements and recruitment.
\end{itemize}
\divider

\myevent{Invited Fellow}{\href{https://www.ipam.ucla.edu/}{IPAM - UCLA}}{Sep 2017 -- Dec 2020}{Los Angeles (USA)}
\begin{itemize}
\item Long research program: \href{https://www.ipam.ucla.edu/programs/long-programs/complex-high-dimensional-energy-landscapes/}{\underline{Complex High-Dimensional Energy Landscapes}}
\item Engaged in applied mathematics \& multi-disciplinary team work
\end{itemize}
\divider

\myevent{Research Assistant}{University of Basel}{Aug 2016 -- May 2018}{Basel (CH)}
\begin{itemize}
\item Integrated theoretical chemistry and machine learning%, achieved data driven prediction of electronic properties properties.
\item Managed multiple projects with a focus on strategic prioritisation%, bridging team communication gaps
\item Organising Committee: \href{https://postdocretreat.biozentrum.unibas.ch/?page_id=1677}{\underline{\emph{The 2017 Basel Postdoctoral Network Retreat}}}
%\item Lab assistant - Infrared (IR) Spectroscopy 
\end{itemize}
\divider

\myevent{Scientific Collaborator}{University of Liege}{Sep 2015 -- Dec 2022}{Liege (Be)}
\begin{itemize}
%\item Drove the advancement in the study of fundamental interactions affecting material properties
\item Provided new insights into temperature-dependent material behaviour
%\item Expertise in harmonising theoretical predictions with experimental data
%\item Grant Management: 
\end{itemize}
\divider

\myevent{PhD Student (FRIA-FNRS personal grant)}{Li\`ege University - Nanomat}{Mar 2011 -- Aug 2015}{Liege (Be)} 
\begin{itemize}
\item Project management: running research grant of \euro{10k} over 4 years
%Modelling temperature-dependent magnetic effects 
\item Research in computational solid state physics: Quantum transport, Spintronics, nano-devices, modelling of complex phenomena
%\item Software development in \href{https://www.abinit.org}{\underline{ABINIT}}
\item Visiting student at The University of Texas at Austin (USA)
\end{itemize}
%\divider

%\newpage
\cvsection{Projects}

\myproject{ML for Spectroscopy}{Atom \& EUSpecLab}{Sep 2022 -- Feb 2024}
\begin{itemize}
\item Application of ML Canonical Sampling in MoSx bi-layer materials 
\item Application of DNN methods to predict IR spectra of Cu clusters
\item ML interatomic potentials for transport quantities of Si nanolayers
\end{itemize}
\divider

\myproject{Batteries}{Toyota Motor Europe \& Atom}{Sep 2020 -- Feb 2024}
\begin{itemize}
\item Electrothermal simulations to understand battery stack for automotive
\item Transport properties of Li-O2 electrolysers by MD and ML models
\item Benchmark of ML Interatomic Potentials and classical force fields % (\color{red}{: network based and GAN models}) 
%\item Model to compare theoretical viscosity to measured leakage over 1000 different polymeric electrolysers 
\item Active Learning prediction of polymeric electrolysers degradation
\end{itemize}
\divider

\myproject{Fuel cells}{Toyota Motor Europe \& Atom}{Sep 2019 -- Feb 2024}
\begin{itemize}
\item Application-driven material design for thermodynamical H$_2$ storage
\item Developed a suite of workflows to manage vast amount of simulations
\item Simulated $\sim$1M nano-structures for room temperature absorption
\item Computational pipeline to simulate $\sim$150k Metal Organic Frameworks
\item Feature Selection \& Genetic Algorithm for H$_2$ adsorption into crystals 
\item Calculated stress-strain resistance of carbon nano tubes and enhanced resistance with cross-functional linking for high pressure H$_2$ storage
\item Post DFT methods to design semiconductor for water photo-catalysis 
\end{itemize}
\divider

\myproject{Gas exhaust catalysis}{Toyota Motor Europe}{Sep 2021 -- Sep 2023}
\begin{itemize}
\item Clustering model to translate over 1M chemical hydrocarbon reactions into simplified (5-10), efficient model for human analysis
\end{itemize}
\divider

\myproject{Lubricants}{Toyota Motor Europe}{Jun 2018 -- Aug 2021}
\begin{itemize}
\item Explained tribological macroscopic effects from molecular interactions
\item Benchmark of several physics-inspired ML models, Feature Selection
\end{itemize}
\divider

\myproject{\href{https://www.ipam.ucla.edu/programs/long-programs/complex-high-dimensional-energy-landscapes/}{\underline{\emph{Energy Landscapes}}}}{IPAM}{Sep 2017 -- Dec 2020}
\begin{itemize}
\item Bayesian optimisation for fast geometry optimisation in DFT
\item DCNN recognition of molecular interaction from electronic structure
\end{itemize}
\divider

\myproject{\href{https://nccr-marvel.ch/}{\underline{\emph{NCCR MARVEL}}}}{UBasel}{Aug 2016 -- May 2018}
\begin{itemize}
\item Machine Learning of electronic-structure materials properties
\item QMAT-X: a reference database for crystallographic Machine Learning 
%\item Kernel Ridge Regression analysis of chemical compound space
\item Feature Analysis and Algorithms benchmark (KRR, RF) for Quantum ML
\end{itemize}
\divider

\myproject{Spin-Caloritronics materials}{ULiege}{Sep 2011 -- Apr 2020}
\begin{itemize}
\item Ab-initio study of electron--phonon coupling in metals
\item Temperature dependence of spin-wave propagation stiffness
\item Interaction of magnetic and vibrational perturbations in materials 
\end{itemize}
\divider

\myproject{The \href{https://www.abinit.org}{\underline{\emph{ABINIT}}} software package}{ULiege}{Sep 2011 -- Dec 2015}
\begin{itemize}
\item Collaborated with a global team to an open-source software package
\item Parallelisation of phonon calculations on independent k-points
\item Analysis and verification to ensure versions consistency
\end{itemize}
\divider
 
\myproject{High-pressure phase transitions}{ULiege}{Mar 2011 -- Dec 2011} 
\begin{itemize}
\item Explained unusual crystallographic phase transition of Calcium via DFT
\end{itemize}
\divider

\myproject{Spin Glasses for frustrated systems}{UBari}{Mar 2010 -- Sep 2010} 
\begin{itemize}
\item Analytical method to calculate free energy of frustrated systems 
\end{itemize}
%\divider

 %% LEFT COLUMN ENDS
\switchcolumn

\cvsection{Hard skills}

\cvtag{R\&D}
\cvtag{Big Data}
\cvtag{Machine Learning}
\cvtag{AI}
\cvtag{B2B}
\cvtag{Pipelines}
\cvtag{Data Science}
\cvtag{Algorithms}
\cvtag{Modelling}
\cvtag{Data Analysis}
\cvtag{Statistical Analysis}
\cvtag{ETL}
\cvtag{EDA}
\cvtag{Data Visualisation}
\cvtag{Optimisation}
\cvtag{High Performance Computing}\\
\cvtag{Software Development}
\cvtag{Python}
\cvtag{Pandas}
\cvtag{Numpy}
\cvtag{Scipy}
\cvtag{Pydantic}
\cvtag{ScikitLearn}
\cvtag{Neural Networks}
\cvtag{C++}
\cvtag{Bash}
\cvtag{Databases}
\cvtag{MongoDB}
\cvtag{SQL}
\cvtag{Git \& Version Control}
\cvtag{CI/CD}
\cvtag{Cloud Computing}
\cvtag{GCP}
\cvtag{AWS}
\cvtag{Multiscale Simulations}
\cvtag{DFT}
\cvtag{MD}
\cvtag{Monte Carlo}
\cvtag{Quantum Computing}
\cvtag{Time Series Analysis}
\cvtag{Forecasting}
\cvtag{Computer Vision}
\cvtag{Numerical Methods}
\cvtag{Energy Materials}
\cvtag{Automotive}


\cvsection{Soft Skills}
\cvtag{Dynamic}
\cvtag{Target oriented}
\cvtag{Flexible}\\
\cvtag{Analytical thinking}
\cvtag{Problem solving}
\cvtag{Attention to details}
\cvtag{Team work}
\cvtag{Abstraction}
\cvtag{Passion}
\cvtag{Commitment}
%\cvtag{Coordination}
\cvtag{Supervision}
\cvtag{Reliable}
\cvtag{Leadership}
\cvtag{Time Management}
\cvtag{Communication}
\cvtag{Rigorous}
\cvtag{Versatile}
\cvtag{Storytelling}
\cvtag{Divulgation}
\cvtag{Proof of Concepts}


\cvsection{Languages}
\cvskill{Italian - Native}{5}
\cvskill{English}{5}
\cvskill{French}{5}
\cvskill{Spanish}{3}
\cvskill{German}{2}
\cvskill{Dutch}{1}
\cvskill{Portuguese}{1}

\newpage

\cvsection{Referees}
\begin{itemize}
\item Dr. Konstantinos Gkagkas \hfill \href{mailto:konstantinos.gkagkas@toyota-europe.com}{\faEnvelope}
\item[] Manager, Toyota Motor Europe 
\item Prof. Anatole von Lilienfeld\hfill \href{mailto:anatole.vonlilienfeld@utoronto.ca}{\faEnvelope}
\item[] Lab head: \href{https://chemspacelab.chem.utoronto.ca}{\underline{Chemspacelab}}
\item Prof. Matthieu Verstraete \hfill \href{mailto:Matthieu.Verstraete@uliege.be}{\faEnvelope}
\item[] Ph.D. supervisor, ULiege
\item[] Lab head: \href{http://www.nanomat.ulg.ac.be/}{\underline{Nanomat}}
\end{itemize}


\cvsection{Certificates}{
\begin{itemize}
\item \href{https://xpro.mit.edu/certificate/b8e20425-0c57-4a63-939c-ab75ccf4fdff/}{Quantum Computing} \hfill2024
\item Introduction to MongoDB \hfill2020
\item Linux admin \hfill2019
\item Introduction to SQL \hfill2018

\end{itemize}
}

\cvsection{Publications}
\begin{enumerate}
\item \href{https://iopscience.iop.org/article/10.1088/1367-2630/accca1/meta}{New Journal of Physics \textbf{25}, 2023}
\item \href{https://pubs.rsc.org/en/content/articlelanding/2022/gc/d2gc01567f/unauth}{Green Chem. \textbf{24}, 2022}
\item \href{https://www.sciencedirect.com/science/article/abs/pii/S0360319921021716?via\%3Dihub}{Int. J. Hydrog \textbf{021}, 27612, 2021}
\item \href{https://journals.aps.org/prb/abstract/10.1103/PhysRevB.102.155128}{Phys. Rev. B \textbf{102}, 155128, 2020}
\item \href{https://link.aps.org/doi/10.1103/PhysRevB.97.214417}{Phys. Rev. B \textbf{97}, 214417, 2018}
\item \href{http://www.ipam.ucla.edu/wp-content/uploads/2018/10/Energy-Landscapes-2017-White-Paper.pdf}{Complex Energy Landscapes, 2017}
\item \href{https://www.sciencedirect.com/science/article/pii/S0010465516300923}{Com. Phys. Comm. \textbf{205}, 106, 2016}
\item \href{https://link.aps.org/doi/10.1103/PhysRevLett.111.025503}{Phys. Rev. Lett. \textbf{111}, 2013}
\end{enumerate}

\cvsection{Education}

\begin{itemize}

\item PhD Computational Physics\hfill 2015
\item[] Liege University (Be)

\item Visiting PhD student \hfill 2013
\item[] Austin University (TX/USA)

\item MS Theoretical Physics\hfill 2010
\item[] Bari University (It) %\hfill 110/110 %{cum laude}   
%\item[] \href{http://cdlfbari.cloud.ba.infn.it/wp-content/uploads/file-manager/CIF/Magistrale/Tesi\%20di\%20laurea/09-10-DI\%20GENNARO\%20Marco.pdf}{\underline{Replica Trick in a finite size spin glass}}
%\item[] Random Matrices in Statistical Physics

\item BS General Physics\hfill 2008
\item[] Bari University (It)  %\hfill 110/110  
%\item[] \href{http://cdlfbari.cloud.ba.infn.it/wp-content/uploads/file-manager/CIF/Triennale/Tesi\%20di\%20laurea/08-09-DIGENNARO\%20Marco.pdf}{\underline{L'evoluzione simulata: vita artificiale}}
%\item[] \href{http://cdlfbari.cloud.ba.infn.it/wp-content/uploads/file-manager/CIF/Triennale/Tesi\%20di\%20laurea/08-09-DIGENNARO\%20Marco.pdf}{\underline{algoritmi genetici}}

\item Erasmus project \hfill 2008
\item[] Sorbonne University/CNRS (Fr)
%\item[] Sorbonne University/CNRS $\cdot$ Introduction \`a la m\'ethode Monte Carlo% - Exemples de simulation num\'erique en physique statistique et en dynamique d’interaction des \'electrons dans la mati\'ere
\item Liceo scientifico \hfill 2005
\item[] Ruvo di Puglia (It) %\hfill 95/100

\end{itemize}

\cvsection{Awards}{
\begin{itemize}
\item Finalist MT180 \hfill2015
\item FWB travel research grant \hfill2013
\item FRIA research fellowship \hfill2011
\item MS Committee award \hfill2010

\end{itemize}}

\cvsection{LEISURES activities}
\begin{itemize}
\item Ultimate frisbee, rock climbing
\item Social dance: Lindy Hop
\item Outdoor activities: bicycling, hiking 
\item Language tandem, Chess
\end{itemize}

\end{paracol}


\end{document}
